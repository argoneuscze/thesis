% !TeX spellcheck = en_GB

\documentclass[thesis=M,english,hidelinks]{FITthesis}[2018/03/12]

\usepackage[utf8]{inputenc}

\usepackage{hyperref}

\usepackage{graphicx}
\graphicspath{{./figures/}}

\usepackage{dirtree}

\usepackage{amsmath} % maths
\DeclareMathOperator{\arctantwo}{arctan2}
\DeclareMathOperator{\sgn}{sgn}

\usepackage[]{algorithm2e}

\usepackage{listings}

\department{Department of Theoretical Computer Science}
\title{Exploring use of non-negative matrix factorization for lossy audio compression}
\authorGN{Tomáš} %author's given name/names
\authorFN{Drbota} %author's surname
\author{Tomáš Drbota} %author's name without academic degrees
\authorWithDegrees{Bc. Tomáš Drbota} %author's name with academic degrees
\supervisor{doc. Ing. Ivan Šimeček, Ph.D.}
\acknowledgements{I would like to thank my supervisor for always being quick to consult any arising issues. My thanks also go to my family and friends for supporting me throughout the entire time of my studies.}
\abstractEN{Non-negative matrix factorization has been successfully applied in various scenarios involving analysis of large chunks of data and finding patterns in them for later use. It's used to perform things such as face recognition, source separation or image compression among others.

The purpose of this thesis is to research possible uses of non-negative matrix factorization in the problem of lossy audio compression. A reference audio encoder and decoder using NMF will be implemented and various experiments using this encoder will be conducted. The results will be measured and compared to existing audio compressing solutions.}
\abstractCS{Algoritmus nezáporného rozkladu matic byl úspěsně použit pro mnoho různých problémů, které souvisí s analýzou velkého množství dat a z nich plynoucích souvislostí. Využívá se např. pro rozpoznávání obličejů, separaci signálů či kompresi obrázků.

Cílem této práce je prozkoumat možné využití nezáporného rozkladu matic pro ztrátovou kompresi zvuku. Bude naprogramován referenční kodér a dekodér zvuku využívajicí NMF, a s ním budou potom provedeny různé experimenty. Výsledky budou změřeny a srovnány vůči existujícím nástrojům pro kompresi zvuku.}
\placeForDeclarationOfAuthenticity{Prague}
\keywordsCS{ztrátová komprese, zvuk, komprese dat, nmf, kódování}
\keywordsEN{lossy compression, audio, data compression, nmf, encoding}
\declarationOfAuthenticityOption{5}
% \website{http://site.example/thesis} %optional thesis URL

\begin{document}
	
\addtocontents{toc}{\setcounter{tocdepth}{2}}

\chapter{Introduction}
A large part of modern compilers is composed of various optimizations meant to either make the resulting program run faster, be smaller in size, or a compromise between these two. The problem is - how do you decide which optimizations to use, how, and in which order? Some optimizations may depend on others, some might be mutually exclusive, and some might even change semantics of the program itself in order to achieve faster processing speed. All these considerations have to be made by someone - in most cases the compiler itself.

But in specific scenarios - for example certain architectures or hardware, and even on specific problems, the compiler may not be able to make an optimal decision by itself. Furthermore, certain optimizations may alter the semantics of the program, which could be an unwanted side effect and might lead to unexpected results during execution.

In such cases, it's up to the programmer to specify which optimizations should be run, optionally along with detailed arguments for them. The problem is, there is currently no universal way to specify most of them in the code itself. There are exceptions, such as \texttt{\#pragma loop} attributes (in C/C++ code) to specify loop unrolling options, but currently, the only way to truly fine-tune all the optimizations is to use the built-in \texttt{opt} tool, which only works on a module level, and depending on your needs, its usage can be unnecessarily verbose.

The purpose of this work is to investigate possibilities of extending this functionality into the code itself, and to provide more granularity in its application, i.e. allow the programmer to specify its scope, instead of having to apply it to the entire module.

Thus, the first part of the thesis focuses on introducing the framework itself, namely LLVM and Clang. I will describe how the compiler currently works and what kind of optimizations it is capable of, optionally if there already exists any way to influence them.

The second part will then expand upon this base, and will focus on designing and implementing a general approach to "hint" the compiler what optimizations it should run, in which scope, and with what parameters.

After these improvements are implemented, I'm going to benchmark my code with these new customized optimizations, and compare the results to what the compiler would produce on its own.

\chapter{Digital audio}
Sound as we know it can be defined as a physical wave travelling through air or another means. \cite{you_2010} It can be measured as change in air pressure surrounding an object. Once we have this electrical representation of the wave, we can convert it back and consequently play using speakers.

In the real world, these sound waves are generally composed of many different kinds of waves, with differing frequencies and amplitudes. The human ear can tell the difference between high (whistling) and low frequencies (drums), and knowledge of this will be useful later when we are discussing audio encoding.

- TODO image of audio signal -

\section{Notation}
Below is a summary of the notation this chapter will be using.

\begin{table}[htbp]\caption{Digital audio notation}
\begin{tabular}{r l}
$t$ & symbol representing a time value in seconds \\
$\tau$ & symbol representing a "slow" time, time index with a lower resolution than $t$ \\
$\xi$ & symbol representing a frequency value in hertz \\
$x(t)$ & function representing the amplitude of a continuous signal at a time $t$ \\
$x_n$ & sequence representing the amplitude of a discrete signal indexed by $n$ \\
$w(t)$ & continuous windowing function at a time $t$ \\
$w_n$ & discrete windowing function indexed by $n$ \\
$X(\xi)$ & function representing the frequency component of a signal for a frequency $\xi$ \\
$S(\xi)$ & the Fourier transform of a continuous signal \\
$S_k$ & the discrete Fourier transform of a discrete signal \\
$S(\tau, \xi)$ & the short-time Fourier transform of a continuous signal \\
$S_{k, \xi}$ & the discrete short-time Fourier transform of a discrete signal \\
$M_k$ & the Modified discrete cosine transform of a discrete signal
\end{tabular}
\end{table}

\section{Important terms}
.. TODO ..
sampling
nyquist frequency/limit
quantization
transient
aliasing
spectral leakage
windowing

\section{Digital audio representation}
Most commonly, the amount of air pressure is sampled many times a second and after being processed this information is stored as a discrete-time signal using numerical representations - this is what's known as a \emph{digital audio signal}. This entire process is called \emph{digital audio encoding}.

By sampling the audio signal, we will potentially be losing out on some information, but given a high enough sampling rate, the result will be imperceptible to the human ear. For general purpose audio and music, the standard sampling rate is 48 kHz, alternatively 44.1 kHz from the compact disk era.

Once we have our digital signal, there are two distinct kinds of ways we can represent, or, encode it. Both of them have many different data models for encoding \cite{you_2010}, but in this work I am only going to focus on the most relevant ones.

- TODO what compromises are taken when encoding -

\subsection{Time domain representation}
In the time domain, the signal is simply represented as a function of time, where $t$ is the time and $x(t)$ is the raw amplitude, or air pressure, at that point. \cite{bosi_goldberg_2003}

This is the most straightforward representation since it directly correlates to how the signal is being captured in the first place. However, as we will see later, this format is not ideal for storing audio data with any sort of compression.

\subsubsection{PCM}
In the time domain, the most basic encoding we can use is PCM (Pulse Code Modulation). After sampling a signal at uniform intervals, the discrete values are quantized; that is, each range of values is assigned a symbol in (usually) binary code.

For example using 16-bit signed PCM, each sample will be represented as a 16-bit signed integer, or in the case of multiple channels, N 16-bit signed integers, where N is the amount of channels.

PCM serves as a good base for what we are going to talk about next - Frequency domain representation and encoding.

\subsection{Frequency domain representation}
While it's simple to understand and work with for the computer with samples in the form of a sequence of amplitudes, it's difficult to run any sort of meaningful analysis on such data. To better grasp the structure of the audio we're working with, it would be helpful to be able to decompose it into its basic building blocks, so to speak. And that's where frequency based representation comes in.

The goal here is to represent the signal as not a function of time, but rather a function of frequency $X(f)$. That is, instead of having a simple sequence of amplitudes, we will have information about the magnitude for each component from a set of frequency ranges. This description alone is generally more compact than the PCM representation \cite{bosi_goldberg_2003} on top of providing us with useful information about the signal, so it will serve as a good entry point to our compression schemes.

\subsubsection{Fourier transform}
Fourier transform is the first and arguably the most used tool for converting a signal from a function of time $x(t)$ into a function of frequency $X(f)$.

It is based on the \emph{Fourier series}, which is essentially a representation of a periodic function as the linear combination of sines and cosines. \cite{Shatkay:1995:FTP:864947} However, the main difference is that our function need not be periodic.

The Fourier transform of a continuous signal $x$ is defined as: \cite{Recoskie2014ConstrainedNM}

\begin{align}
S(\xi) = \int_{-\infty}^{\infty}x(t)e^{-2\pi it\xi}dt
\end{align}

If we inspect the formula, we can notice that Fourier transform essentially projects our signal into infinity - this wouldn't be a problem if it was a periodic signal, but sampled audio is generally constrained by time. To prevent spectral leakage, we must window the signal before processing it. \cite{heinzel_2002_windows}

The output is a complex number, which provides us with the means to find the magnitude and phase offset for the sinusoid of each frequency $\xi$.

The Fourier transform can also be inverted, providing us with an easy way to obtain the original signal back from its frequency components. The inverse transform is defined as:

\begin{align}
x(t) = \int_{-\infty}^{\infty}S(\xi)e^{2\pi it\xi}d\xi
\end{align}

However, seeing as our samples are discretely sampled, we will need to modify our transform accordingly.

The discrete Fourier transform of a discrete signal $x_0, x_1, ..., x_{N-1}$ is: \cite{Recoskie2014ConstrainedNM}

\begin{align}
S_k = \sum_{n=0}^{N-1}x_ne^{-2\pi ikn/N}
\end{align}

And our inverse is:

\begin{align}
x_n = \frac1N \sum_{k=0}^{N-1}S_ke^{2\pi ikn/N}
\end{align}

The issue is, due to the nature of this process, if we run the Fourier transform on our whole signal, we will only be able to analyse it as a whole, e.g. we won't be able to tell which parts of for example a song are quiet or if there are any parts with very high frequencies - we lose our temporal data.

To alleviate this problem, we can run Fourier transform on smaller chunks of the signal, analyse them separately and later join them back into the original signal. That is the essence of the Short-time Fourier transform.

\subsubsection{Short-time Fourier transform}
When using Short-time Fourier transform, or STFT for short, we first split the signal into smaller segments of equal size and then run Fourier transform on those separately. As such, our output can be projected into two dimensions - specifically a frequency spectrum as a function of time, a spectrogram.

.. TODO picture of spectrogram ..

Doing it this way will let us see how the frequency components change over time instead of taking the spectrum of the entire signal.

As with regular Fourier transform, we'll need to window each segment of the signal, but there is a caveat. Since we have windowed segments, we may be losing some information at the edge of each segment leading to artifacts, and furthermore we may be losing information about transients. To solve this, we'll need to introduce overlapping windows - however, having an overlap will increase the amount of coefficients required.

The continuous version is defined as: \cite{Recoskie2014ConstrainedNM}

\begin{align}
S(\tau, \xi) = \int_{-\infty}^{\infty}x(t)w(t-\tau)e^{-2\pi it\xi}dt
\end{align}

where $w$ is the window function.

But again, as we have discrete samples, we will need to use a discrete short-time Fourier transform, specifically:

\begin{align}
S_{k, \xi} = \sum_{n=-\infty}^{\infty}x_nw_{n-k}e^{-2\pi i\xi n}
\end{align}

And similarly to the regular Fourier Transform, short-time Fourier Transform is also invertible. \cite{selesnick_2009}

STFT is commonly used for audio analysis (TODO source) but in this case it will be used as a means for our NMF compression.

\subsubsection{Modified discrete cosine transform}
Modified discrete cosine transform, or MDCT for short, has become the dominant means of lossy high-quality audio coding. \cite{wang_vilermo_2012_mdct}

It is what's known as a \emph{lapped transform}. This means that when transforming a block into its MDCT coefficients, the basis function overlaps the block's boundaries. \cite{Malvar:1992:SPL:531523} In practice, what this means is that while we have blocks with overlapping windows as in the short-time Fourier transform, the number of coefficients remains the same as without while retaining the relevant properties.

As the name suggests, MDCT is based on the Discrete cosine transform, namely \emph{DCT-IV}, where the main difference is the addition of lapping mentioned above.

What makes MDCT simpler to work with compared to Fourier transform is that not only do we not need more coefficients despite overlapping, they are also real numbers as opposed to complex numbers, lowering the amount of bytes necessary to store them.

It is a linear function $f: \mathbf{R}^{2N} \rightarrow \mathbf{R}^N$, defined as: \cite{Babu2013FastAE}

\begin{align}
M_k = \sum_{n=0}^{N-1} x_n \cos \left\lbrace \frac{(2n+1+ \frac{N}{2} )(2k+1)\pi }{2N} \right\rbrace
\end{align}

for $k = 0, 1, \ldots, \frac{N}{2}-1$.

It is assumed that $x(n)$ is already windowed by an appropriate windowing function $w$.

MDCT is also invertible, and its inversion is defined as:

\begin{align}
\bar{x}_n &= \sum_{k=0}^{\frac{N}{2}-1} M_k \cos \left\lbrace \frac{(2n+1+ \frac{N}{2} )(2k+1)\pi }{2N} \right\rbrace
\end{align}

for $n = 0, 1, \ldots, N-1$.

It's important to note that the inverted transformed sequence $\bar{x}_n$ by itself does not correspond to the original signal $x_n$ \cite{prince_1986_tdac_1}. To achieve perfect invertibility, we must add subsequent overlapping blocks of the inverted MDCT (IMDCT). This method is called \emph{time domain aliasing cancellation} \cite{prince_1986_tdac_2}, or TDAC for short. As the name suggests, it mainly helps remove artifacts on the boundaries between transform blocks.

\section{Psychoacoustics}
Apart from time-frequency representations being generally more compact, they also give us the ability to analyse, isolate or modify the frequency composition of a given signal. This comprises a large chunk of the audio compressing process.

The field of psychoacoustics studies sound perception - that is, how our ears work and how we perceive different kinds of sounds. There are many different characteristics to sound that need to be taken into account for a proper psychoacoustic analysis \cite{olson1967music}, split into several categories, namely:

\begin{description}
	\item[tonal] includes pitch, timbre, melody harmony
	\item[dynamic] based on loudness
	\item[temporal] involves time, duration, tempo and rhythm
	\item[qualitative] represents harmonic constitution of the tone
\end{description}

For music, it's important to balance these four qualities appropriately. For compression, the most important qualities for us in scope of this work are going to be tonal (pitch) and dynamic (loudness).

\subsection{Pitch}
Pitch is a characteristic that comes from a frequency. The difference between the two is that pitch is our subjective perception of the tone whereas a frequency is an objective measure. Despite this fact, pitch is often quantified as a frequency using Hertz as its unit.

The lower bound of human hearing is around 20 Hz whereas the upper bound is most commonly cited as 20 000 Hz, or 20 kHz. \cite{rosen1993hearing} In a laboratory environment, people have been found to hear as low as 12 Hz. As people age, our hearing gets progressively worse and a healthy adult younger than 40 years can generally perceive frequencies only up to 15 kHz. \cite{olson1967music}

The human ear is capable of distinguishing different frequencies fairly accurately, though accuracy gets lower with increasing frequency. It's easier for our ears to tell a difference between 500 Hz and 520 Hz compared to the difference between 5000 Hz and 5020 Hz. \cite{smacdon_2018}

Furthermore, if we hear two different tones simultaneously, but their frequencies are close enough to one another, we may perceive them as a combination of tones rather than separate tones. Frequency ranges, or bands, where this phenomenon happens, are called \emph{critical bands}. \cite{fletcher_1940} It's also possible for one tone to mask the other entirely, and then we get what's called \emph{auditory masking}. \cite{gelfand1990hearing}

Based on the knowledge of the existence of these critical bands, it's possible to devise a system that specifies the range of each band in human hearing. One such scale that is commonly used is called the \emph{Bark scale}.

\subsubsection{Bark scale}
The Bark scale ranges from 1 to 24 Barks, where each Bark corresponds to a single critical band of human hearing. \cite{fastl_2006} The perceived difference in pitch between each band should be the same, despite the scale not growing linearly in terms of frequency ranges. Specifically, until around 500 Hz, the scale is roughly linear, but above that it has a more logarithmic growth. \cite{hermes_filter}

The Bark scale is commonly used as reference for audio encoding codecs, as we will see later. Knowledge of these critical bands allows for more educated byte allocation during the quantization process when compressing a frequency domain representation.

.. TODO image/table of scale ..

\subsection{Loudness}
What people often decide as loudness is really called \emph{sound pressure level} and it's measured in decibels (dB), however it has some shortcomings when it comes to psychoacoustic analysis.

It is defined as following: \cite{behar_1984}

\begin{align}
L_p = 20 \log_{10} \left( \frac{p}{p_0} \right) \text{dB}
\end{align}

where $p$ is a sound's sound pressure and $p_0$ is a reference sound pressure, also called the threshold of human hearing.

While this metric is very popular, it doesn't account for the fact that different frequencies have a different perceived loudness for a person's ears. \cite{olson1967music} There is a lot of research in recent years into how different frequencies impact our perception and hearing \cite{kuwano_1989}, but that is out of scope of this work. For more information about the exact definitions of loudness, refer to \cite{olson1967music}.

\subsection{Auditory masking}
As mentioned above, when it comes to audio masking, and therefore audio compression, we must not only take into account the critical bands as per e.g. the Bark scale, but also their intensity.

For example a lower frequency sound may mask one of a higher frequency, but the other way around does not apply. \cite{gelfand1990hearing} Modern audio encoders take this into account and using this knowledge are able to eliminate sounds that exist in the original signal, but are not perceivable by humans.

There are two important different kinds of masking effects - \emph{simultaneous} masking and \emph{temporal} masking. \cite{Raissi2002TheTB}

Simultaneous masking is what I have hinted at above - when there are two sounds within the same critical band, the dominant one may mask other frequencies within the same band.
.. TODO image ..

Temporal masking does not occur in the frequency domain, but the time domain. The essence is that a stronger tonal component may mask a weaker one if they appear within a small window of time in succession.
.. TODO image  ..







\chapter{Non-negative matrix factorization}
In today's age of big data, machine learning and various other fields, it's important to have ways to quickly analyse these datasets and ideally find patterns within. Non-negative matrix factorization is one of the paradigms suitable for that task.

In layman's terms, what NMF does is that when we are given a large set of data, for example a matrix representing books and their review scores from people, we can extract certain hidden "features" from it using NMF, in this case representing e.g. various genres (basis matrix) and how prominent they are in a given book (weight matrix). And then, using these two matrices, we are able to estimate or even predict what kind of books a user would like - this is a simple example of a possible recommendation algorithm.

\section{Linear dimensionality reduction}
Non-negative matrix factorization, or NMF, falls under \emph{linear dimensionality reduction} techniques. These are used widely for noise filtering, feature selection or compression, among others.

LDR can be defined as following: \cite{nmf_why_how}

\begin{align}
x_j &\approx \sum_{k=1}^{r}w_kh_j(k) &\text{for some weights $h_j \in \mathbf{R}^r$}
\end{align}

where given a data set of size $n$, we define $x_j \in \mathbf{R}^p$ for $1 \leq j \leq n$, $r < \min(p,n)$, and $w_k \in \mathbf{R}^p$ for $1 \leq k \leq r$.

What this effectively means is that we represent $p$-dimensional data points in a $r$-dimensional linear subspace, with basis elements $w_k$ and data coordinates given by vectors $h_j$. LDR defined in this manner is equivalent to low-rank matrix approximation, which is the essence of non-negative matrix factorization.

\section{NMF definition}
Non-negative matrix factorization solves the following NP-hard problem:

Given a non-negative matrix $V$, find non-negative matrix factors $W$ and $H$ such that:

\begin{align}
V \approx WH
\end{align}

That is, given a set of multivariate $n$-dimensional data vectors, we place these vectors in the columns of a $n \times m$ matrix $V$, where $m$ is the amount of examples we have. We then approximately factorize this matrix into two different matrices: a $n \times r$ matrix $W$ and a $r \times m$ matrix $H$. We generally choose $r < \min(n,m)$ (though this is not required) so that the two matrices are smaller than the original matrix $V$, essentially compressing it. \cite{nmf_algorithms}

\section{Classification}
NMF is as of currently still a relevant research topic, and has been explored by researchers from many different fields including mathematicians, statisticians, computer scientists or biologists. Given the wide range of use, over time it lead to different variations and additional constraints on the algorithms. Therefore, a taxonomy system was proposed in \cite{wang_zhang_2013}, outlined below.

\begin{figure}[ht]
	\caption[NMF classification]{The NMF classification as per \cite{wang_zhang_2013}.}
	\centering
	\includegraphics[width=\textwidth]{nmf_classification.png}
\end{figure}

\subsection{Basic NMF}
This is the basic model which only enforces non-negativity, and which all the following ones build upon.

However, due to its unconstrained nature, without any other constraints there are many possible solutions  which may lead to the algorithm's performance to vary. Further constraints outlined below help in the search of unique solutions and optimizing for specific scenarios.

\subsection{Constrained NMF (CNMF)}
Constrained NMF imposes additional constraints on the resulting matrices, namely:

\begin{description}
	\item[Sparse NMF] SPNMF, sparseness constraint
	\item[Orthogonal NMF] ONMF, orthogonality constraint
	\item[Discriminant NMF] DNMF, couples discriminant information along with the decomposition
	\item[NMF on manifold] MNMF, preserves local topological properties
\end{description}

\subsection{Structured NMF (SNMF)}
Structured NMF modifies standard factorization formulations:

\begin{description}
	\item[Weighed NMF] WNMF, attaches weights to different elements relative to their importance
	\item[Convolutive NMF] CVNMF, considers time-frequency domain factorization
	\item[Non-negative Matrix Trifactorization] NMTF, decomposes the data into three matrices
\end{description}

\subsection{Generalized NMF (GNMF)}
Generalized NMF can be considered a broader variant of Basic NMF, where conventional data types or factorization modes may be replaced with something different. It's split as follows:

\begin{description}
	\item[Semi-NMF] relaxes the non-negativity constraint on a specific factor matrix
	\item[Non-negative Tensor Factorization] NTF, generalizes the model to higher dimensional tensors
	\item[Non-negative Matrix-set Factorization] NMSF, extends the data sets from matrices to matrix-sets
	\item[Kernel NMF] KNMF, non-linear model of NMF
\end{description}

\section{Properties}
The additional constraint of non-negativity is important, as results show that it leads to a natural higher sparseness in both the basis matrix ($W$) and the encoding matrix ($H$). Additionally, non-negativity leads to a parts-based representation, which is similar to how our brains are presumed to work, basically combining parts in an additive manner to form a whole instead of subtracting. \cite{nmf_parts_objects} This sparseness makes it even easier to further compress the resulting matrices, saving us more space.

However this isn't without any downsides. While the concept of adding parts together seems to make a lot of sense, there is an issue. Since NMF employs a holistic approach, the additive parts learned by it in an unsupervised mode only considers features on a global level, and does not allow for representation of spatially localized features. \cite{li_spatial_lnmf_2001}

So while on paper NMF might seem better than PCA or SVD for a parts-based representation, it only comes at a cost of increased complexity, and since both PCA and SVD have a more compact spectrum than NMF, we must consider if this is worth the trade-off. \cite{wang_zhang_2013}

\section{Algorithms}
For the purposes of this thesis, we will only consider algorithms for Basic NMF. While NMF has been widely used in sound analysis etc. as we'll see below, its use for audio compression specifically is rare and there are limited resources to provide insight into utilizing possible constraints, therefore we will only be using the standard version.

Finding a decomposition of a matrix $V$ into matrices $W$ and $H$ is an NP-hard problem, and as such, the resulting matrices are generally only approximated over a number of iterations of an optimization algorithm. What this means in practice is that it's likely a result we'll find is sub-optimal or a local minimum.

\subsection{Cost function}
When using iterative updates, in each step of the process we need to evaluate the quality of the approximation. The function that does this is called the \emph{cost function}, or \emph{objective function}.

There are two simple commonly used functions. Firstly, we can use squared Euclidean distance: \cite{pentti_pmf_1997}

\begin{align}
||A-B||^2 = \sum_{ij}(A_{ij} - B_{ij})^2
\end{align}

This is lower bounded by 0, which it only is equal to if $A = B$.

Another metric we can use is based on Kullback-Leibler divergence, and is defined as such: \cite{nmf_algorithms}

\begin{align}
D(A||B) = \sum_{ij} \left( A_{ij} \log \frac{A_{ij}}{B_{ij}} - A_{ij} + B_{ij} \right)
\end{align}

\subsection{Update rules}
With the cost function in place, we now need a function to apply each iteration to try and minimize the value of the cost function. It has been found that a good compromise between speed and ease of implementation is to use what's called \emph{multiplicative update rules}. \cite{nmf_algorithms} Despite being over 15 years old, they are still very commonly used exactly for this reason.

For non-increasing Euclidean distance $||V - WH||$, if $W$ and $H$ are at a stationary point of distance, we may use these rules:

\begin{align}
H_{a \mu} & \leftarrow H_{a \mu} \frac{(W^TV)_{a \mu}}{(W^TWH)_{a \mu}} \\
W_{ia} & \leftarrow W_{ia} \frac{(VH^T)_{ia}}{(WHH^T)_{ia}}
\end{align}

And for non-increasing divergence $D(V||WH)$, if $W$ and $H$ are at a stationary point of divergence, we can use this:

\begin{align}
H_{a \mu} & \leftarrow H_{a \mu} \frac{\sum_i W_{ia} V_{i \mu} / (WH)_{i \mu}}{\sum_k W_{ka}} \\
W_{ia} & \leftarrow W_{ia} \frac{\sum_\mu H_{a \mu} V_{i \mu} / (WH)_{i \mu}}{\sum_v H_{av}}
\end{align}

\subsection{Initialization}
Before we can use our update rules to iteratively optimize the cost function, we need some initial value to initialize the matrices $W$ and $H$ to, first. In practice, different initializations generally yield different solutions, so this is worth considering. \cite{naik_2015_nmf_advances}

The most basic way of initialization is to simply initialize the matrices with random values. This generally works decently but you somewhat lose out on controlling the composition of the matrices. A small way to improve this is to generate random matrices a couple times and pick the one with the lowest cost function value, but the issue remains.

As per \cite{naik_2015_nmf_advances}, there are tons of different ways to initialize the matrices, usually relating to constraints on the matrices, e.g. initialize $H$ in such a way that none of the values are above a certain threshold etc. But given the low amount of research on audio compression using NMF, it's difficult to gauge which of these might work better for audio than others, and as such this work will not elaborate on different ones further.

\section{Use in digital audio processing}
In digital audio, Non-negative matrix factorization is mostly used as a tool for analysis rather than compression. The ability to extract hidden features from a given signal is useful for certain things.

For example NMF sees use in audio separation tasks. \cite{fevotte_audio_separation_2017} The gist of this lies in first creating a spectrogram of the signal using STFT and then using NMF decomposition to isolate different kinds of sounds or instruments from it, roughly represented by the basis matrix.

Another example of use is musical transcription. \cite{recoskie_mann_2014} The idea here is to process an input signal via STFT, further filtering and NMF, to isolate individual notes from e.g. a piano piece.

Both of these methods have seen some success, but as we will see later, applying NMF to compression rather than analysis is not a simple task and might not even be worth it.




\chapter{Data compression}
\input{chapters/data_compression}

\chapter{Design}
In this chapter I will give an overview of how this audio codec will be designed, and explaining the various decisions I made along the way.

I have decided to call this new codec \emph{ANMF} (stands for Audio-NMF), using files with the extension \emph{.anmfx}, where \emph{x} represents the compression method. It will be implemented as a command line utility.

There are currently three different ANMF formats that you can choose from, and their main difference is which audio representation is being compressed by NMF. They are as follows:

\begin{description}
	\item[ANMF-RAW] denoted by \emph{r}, compresses the signal in PCM form (time domain)
	\item[ANMF-MDCT] denoted by \emph{m}, compresses the signal transformed with MDCT (frequency domain)
	\item[ANMF-STFT] denoted by \emph{s}, compresses the signal transformed with STFT (frequency domain)
\end{description}

\section{WAVE file}
WAVE, WAV or Waveform audio is a file format for storing digitized audio, created as a joint design by the Microsoft Corporation and the IBM Corporation. They are built on top of the chunk-based RIFF format. For details on the specific structure of a WAVE file please refer to \cite{sapp_pcm}.

It stores raw uncompressed audio samples in PCM format along with some metadata and will serve as both the standard input and output to the ANMF codec. Most commonly used formats can be converted to and from WAV as well and thus it will serve as a good baseline.

Samples in this format can be represented by different datatypes, I chose 16-bit signed integers, i.e. each sample's amplitude is represented by a whole number between $-32768$ and $32767$. The sample rate can vary, but a good standard value the experiments will use is $44.1$ kHz, which corresponds to audio CD quality.

\section {ANMF File structure}
The base container for the compressed ANMF file is the same no matter which encoding method you use. The bytes are saved in little endian byte order.

Please refer to Table \ref{tab:anmf_file} and each encoding method's table for the specific file structure. The first eleven bytes are mostly set in stone other than the method specification, but after that it varies greatly.

\begin{table}[htbp]\caption{ANMF file structure}
	\label{tab:anmf_file}
	\centering
	\begin{tabular}{|c|c|l|}
		\hline
		Bytes & Data type & Description \\ \hline
		0-3 & char[] & identifier string "ANMF" \\
		4 & char & method used, can be 'S', 'R' or 'M' \\
		5-6 & uint16 & \# of channels \\
		7-10 & uint32 & sample rate \\
		11-? & enc\_data & encoded data depending on the method \\
		\hline
	\end{tabular}
\end{table}

When serializing matrices to a file, the structure in Table \ref{tab:anmf_serial_matrix} will be used, denoted by a "matrix($dt$)" datatype in the following tables, where $dt$ stands for the datatype used for the matrix elements.

\begin{table}[htbp]\caption{Serialized matrix structure}
	\label{tab:anmf_serial_matrix}
	\centering
	\begin{tabular}{|c|c|l|}
		\hline
		Bytes (relative) & Data type & Description \\ \hline
		0-3 & uint32 & amount of rows in the matrix \\
		4-7 & uint32 & amount of columns in the matrix \\
		8-($8+x-1$) & $dt$ & row-wise values of the matrix, $x = rows*columns$ \\
		\hline
	\end{tabular}
\end{table}

\section{Encoder}
The encoder is responsible for taking a raw audio file and encoding the data within, producing a compressed version of the original. Please refer to Figure \ref{fig:design_encoder} for a visual representation of the process.

\begin{figure}[ht]
	\caption[Encoder overview]{A high level overview of the ANMF audio encoder.}
	\label{fig:design_encoder}
	\centering
	\includegraphics[width=\textwidth]{design_encoder.png}
\end{figure}

Next, each format's encoding process will be outlined (third step in the figure). If the audio file has multiple channels, this process is repeated on each channel separately.

\subsection{ANMF-RAW}
\begin{figure}[ht]
	\caption[ANMF-RAW Encoder]{The encoding scheme for ANMF-RAW.}
	\label{fig:encoding_nmf_raw}
	\centering
	\includegraphics[width=\textwidth]{nmf_raw.png}
\end{figure}

ANMF-RAW works on the principle of applying NMF directly to the PCM audio samples $x_n$. However, the samples are initially an array of 16-bit signed integers, and as such, they need to be processed first before NMF can be used.

A chunk shape is specified to determine how many rows and columns each matrix will have before NMF. We choose a target matrix shape of $1152 \times 200$ to match the amount of samples per frame in the other methods.The sample array is then padded with zeroes to ensure there are enough elements at the end of the array to ensure every matrix can have the same shape and number of elements. The amount of padding must be written to the output so that we know the length of the original array when decoding.

Once the array is padded, we iterate over the samples and split them into equal chunks of size $rows*columns$. This array is then "folded" to produce a matrix of the desired shape. We then obtain a matrix of signed integers, so in order to be able to use NMF, we first need to get rid of all the negative values. To do that, we increment each chunk by the absolute value of its smallest element, guaranteeing that the lowest value in the matrix is $\ge 0$.

Once we have this matrix, we proceed by applying NMF on it, obtaining the basis matrix $W$ and coefficient matrix $H$. Lastly, for each chunk, we write the value we incremented the matrix by, and the two decomposition matrices.

\begin{table}[htbp]\caption{ANMF-RAW data structure}
	\label{tab:anmf_raw_file}
	\centering
	\begin{tabular}{|c|c|l|}
		\hline
		Bytes (relative) & Data type & Description \\ \hline
		0-3 & uint32 & amount of zeroes used to pad the samples \\
		4-7 & uint32 & amount of chunks \\
		8-? & data\_chunk[] & NMF-compressed data chunks (refer to Table \ref{tab:anmf_raw_data}) \\
		\hline
	\end{tabular}
\end{table}

\begin{table}[htbp]\caption{ANMF-RAW structure of each data chunk}
	\label{tab:anmf_raw_data}
	\centering
	\begin{tabular}{|c|c|l|}
		\hline
		Bytes (relative) & Data type & Description \\ \hline
		0-7 & float64 & absolute value that the matrix was incremented by \\
		8-? & matrix(float32) & matrix $W$ \\
		?-? & matrix(float32) & matrix $H$ \\
		\hline
	\end{tabular}
\end{table}

\subsection{ANMF-MDCT}
\begin{figure}[ht]
	\caption[ANMF-MDCT Encoder]{The encoding scheme for ANMF-MDCT.}
	\label{fig:encoding_nmf_mdct}
	\centering
	\includegraphics[width=\textwidth]{nmf_mdct.png}
\end{figure}

In ANMF-MDCT, as the name suggests, the PCM input will be transformed using MDCT as per Section \ref{sec:mdct}. Since this is a lapped transform, each transformed block will have a 50\% overlap with the following block. We choose a frame size of $2N = 1152$ and split the signal into blocks of that size, thus each block will contain $N = 576$ coefficients from its own block, and another $576$ from the following one.

The decision to have a block size of $N = 576$ stems from the fact that this will give us the same amount of frequency resolution that e.g. MP3 uses (as seen in Section \ref{sec:mp3}), which proved to be enough for human hearing.

As before, we first pad the signal at the end with zeroes to align it to the desired block size. To prevent loss of data in the first and the last block due to the overlapping, we further pad the signal by an array of zeroes, equal in size to the size of a block, that is $N$ zeroes both at the beginning and the end of the signal.

Then, we apply a windowing function on each block to bring the values near the edges closer to $0$ to help mitigate spectral leakage. We use the MLT window $w_n^M$ as defined in Section \ref{sec:mlt}.

Finally, we apply the MDCT on each of the windowed blocks and obtain a matrix of MDCT coefficients in the form of real numbers.

This matrix is then split into smaller chunks. For example, if the MDCT matrix contains $576$ rows and $1100$ columns, we might split it into submatrices sized $576 \times 200$, with the last one being $576 \times 100$, as no padding is necessary here. This amount of chunks is written to the output. NMF is then ran on each of the chunks separately and the decomposition matrices serialized.

\begin{table}[htbp]\caption{ANMF-MDCT data structure}
	\label{tab:anmf_mdct_file}
	\centering
	\begin{tabular}{|c|c|l|}
		\hline
		Bytes (relative) & Data type & Description \\ \hline
		0-3 & uint32 & amount of zeroes used to pad the samples \\
		4-7 & uint32 & amount of MDCT submatrix chunks \\
		8-? & data\_chunk[] & NMF-compressed MDCT chunks (refer to Table \ref{tab:anmf_mdct_data}) \\
		\hline
	\end{tabular}
\end{table}

\begin{table}[htbp]\caption{ANMF-MDCT structure of each data chunk}
	\label{tab:anmf_mdct_data}
	\centering
	\begin{tabular}{|c|c|l|}
		\hline
		Bytes (relative) & Data type & Description \\ \hline
		0-7 & float64 & absolute value that the matrix was incremented by \\
		8-? & matrix(float32) & matrix $W$ \\
		?-? & matrix(float32) & matrix $H$ \\
		\hline
	\end{tabular}
\end{table}

\subsection{ANMF-STFT}
\begin{figure}[ht]
	\caption[ANMF-STFT Encoder]{The encoding scheme for ANMF-STFT.}		\label{fig:encoding_nmf_stft}
	\centering
	\includegraphics[width=0.6\textwidth]{nmf_stft.png}
\end{figure}

The design of ANMF-STFT is based on the solution suggested in \cite{nikunen_2010} with some changes along with only utilizing open source solutions.

Like with ANMF-MDCT, we choose a frame size of $N = 1152$, leading to a frequency resolution of $576$ bins. We begin by properly padding the signal to $N$ so that it's possible to be split into equal parts. We then use STFT with 50\% overlap and a block size of $N$, which means we end up with twice the coefficients compared to MDCT, but this is not a major issue.

During STFT, we must again window each block, leading to overlapping windows. We use the Hann window $w_n^H$ for this (as defined in Section \ref{sec:hann}).

Once STFT is finished, we end up with a matrix of Fourier transform coefficients in the form of complex numbers. Trying to apply NMF on the complex numbers directly would yield similar results to ANMF-MDCT, so we have to approach this differently.

If we visualise the complex valued elements in the complex plane, we can instead represent each element $z$ as two separate values:

\begin{description}
	\item[magnitude] also called the modulus, geometrically it's the distance from 0
	\item[phase] also called the argument, geometrically it's the angle from the real axis
\end{description}

To obtain the phase $\phi$ of a complex number $z = x + iy$, we can use the following formula:

\begin{align}
\phi(z) = \arg(z) = \arctantwo(y,x)
\end{align}

And to obtain the magnitude $|z|$ of the complex number:

\begin{align}
|z| = \sqrt{x^2 + y^2}
\end{align}

By calculating the magnitude and phase of every element in the STFT matrix individually, we obtain the magnitude spectrogram and the phase spectrogram respectively. We now need to encode both of them individually.

For the phase matrix, my experiments showed that applying NMF on it leads to a very noticeable loss in quality, so instead I opted for a different solution that ultimately ends up saving more space than NMF would.

The phase matrix contains values ranging from $-\pi$ to $\pi$. These values are uniformly quantized into 8 levels as per Section \ref{sec:unif_quant}. Due to the relative frequency of the boundary values $-\pi$ and $\pi$, a mid-tread quantizer is used. These quantized values are then losslessly encoded using Huffman coding using $n_p = 3$ bits per value and written to output.

\begin{figure}[ht]
	\caption[ANMF-STFT quantized phase frequencies]{Frequency of each quantization level in the phase spectrogram.}
	\label{fig:stft_quant_freq}
	\centering
	\includegraphics[width=\textwidth]{stft_phase_quant_freq.png}
\end{figure}

.. TODO magnitude spectrogram encoding ..

\subsubsection{Huffman encoding}
.. TODO Huffman ..

\section{Decoder}
Similar to the encoder, the decoder simply reverses the encoding process as seen in Figure \ref{fig:design_decoder}. As this process is fairly straightforward for each of the methods, it won't be elaborated on further.

\begin{figure}[ht]
	\caption[Decoder overview]{A high level overview of the ANMF audio decoder.}
	\label{fig:design_decoder}
	\centering
	\includegraphics[width=\textwidth]{design_decoder.png}
\end{figure}


\chapter{Implementation}
This chapter explains the specific details of implementing the audio codec outlined in the previous chapter.

The codec was implemented entirely using \emph{Python 3} (x64) \cite{python3_ref} and the source code is available both as part of this thesis and on my personal GitHub profile \href{https://github.com/argoneuscze/AudioNMF}{[link]}.

The codec could be roughly split into three parts:

\begin{description}
	\item[Command line utility] the command line interface (CLI), interacts with the user
	\item[Audio library] a set of classes and functions responsible for reading/writing audio files
	\item[Compression library] a set of classes and functions responsible for encoding and decoding ANMF files
\end{description}

\section{Used 3rd party libraries}
The codec uses several libraries for things where a tried and tested method is both faster and more reliable than implementing my own. A brief description of each of them follows.

\subsection{click}
The \verb|Click| \cite{py_click} library lets you create command line interfaces in a descriptive manner without having to manually read user input. This helps prevent errors regarding unexpected parameters and such. It also automatically generates the help message describing the various options and arguments, all of which makes it suitable for a project like this.

\subsection{pytest}
\verb|pytest| \cite{py_pytest} is a framework for writing and executing tests in Python. To ensure certain functions work correctly and give expected results, several unit tests have been written.

\subsection{dahuffman}
To create Huffman trees and use Huffman coding, a library called \verb|dahuffman| \cite{py_dahuffman} was utilized.

It determines the frequencies either from a dictionary containing $value \rightarrow frequency$ pairs, or it infers them automatically from existing data (as per Section \ref{sec:huffman}). Once the frequencies are determined, the library is capable of losslessly encoding and decoding arbitrary data containing the symbols from the Huffman tree into a serializable byte array.

If needed, the Huffman table including the exact code for each symbol can be printed to validate it works correctly.

A wrapper class was made around this library called \verb|HuffmanCoder|, to facilitate easier usage and to be able to easily separate different coders using varying dictionaries. It has convenience methods \verb|encode_int_matrix| and \verb|decode_int_matrix|.

When encoding using the encoding method, it flattens the matrix into an array, then \verb|dahuffman| encodes the array using the associated Huffman table, and finally the function returns both the bytes and the amount of rows in the original matrix.

Decoding does the reverse - it takes these bytes and the amount of rows, decodes the bytes into an array of values, and then re-shapes the array into the original matrix.

\subsection{SciPy}
\verb|SciPy| \cite{py_scipy} is short for \emph{Scientific Python} and as the name suggests, this library provides the bulk of the mathematical support for the implementation.

\verb|SciPy| as a whole is more of a stack, or software ecosystem, containing various sub-libraries (covered below). However, the \verb|SciPy| library itself provides some algorithms which are used as described later.

\subsubsection{NumPy}
\verb|NumPy| \cite{py_numpy} is part of the \verb|SciPy| stack and its most important feature for us is that it provides us with an easy way to create and manipulate N-dimensional array objects. In our case, for the most part it's 2D matrices.

It has a friendly syntax and all the core functions for creating matrices, mapping functions over them, multiplying them etc., all of which are very important. It also supports complex numbers and operations on them seamlessly.

Even though it is a Python library, a lot of its performance critical parts are written in lower level languages such as C and are optimized for complex calculations.

\verb|NumPy| is heavily utilized in this work, as the problem of NMF mostly concerns 2D matrices and \verb|NumPy| is ideal for working with those.

\subsubsection{Matplotlib}
\verb|Matplotlib| \cite{py_matplotlib} is also part of the \verb|SciPy| stack and it's best described as a Python plotting library, capable of producing various (primarily 2D) plots and diagrams from data.

As it's closely integrated with \verb|SciPy|, it can natively plot data structures created by e.g. \verb|NumPy| without any boilerplate code necessary.

\verb|Matplotlib| is responsible for most of the diagrams visible in this thesis.

\section{Command line utility}
The command line part of the program (and the entry point) was created using the \verb|Click| library.

The CLI lets you specify which kind of compression you want to use, between ANMF-RAW, ANMF-MDCT and ANMF-STFT, defaulting to ANMF-STFT. When decompressing, it automatically determines the compression type from the file's extension.

The bulk of the work of the command line utility is to pass the input file's handle to the audio library and obtain a reference to the parsed audio data. This data is then passed to the compression library along with a reference to the output file handle, and once the target file is compressed or decompressed, the program returns.

\section{Audio library}
The audio library is capable of reading samples from audio files, or creating audio files from samples, in both cases the only supported format right now is 16-bit signed PCM WAV.

To facilitate reading and writing WAV files, I use the methods contained in the
\verb|scipy.io.wavfile| namespace, specifically \verb|scipy.io.wavfile.read| and \verb|scipy.io.wavfile.write|.

When a WAV file is read, its raw samples and metadata are stored in an internal class called \verb|AudioData|. This class can then either call the compression library to save it as a compressed ANMF file, or it may call into \verb|SciPy| again to save it into a WAV file.

\section{Compression library}
The compression library is a collection of three classes:

\begin{description}
	\item[NMFCompressorRaw] class responsible for compressing and decompressing ANMF-RAW data
	\item[NMFCompressorMDCT] class responsible for compressing and decompressing ANMF-MDCT data
	\item[NMFCompressorSTFT] class responsible for compressing and decompressing ANMF-STFT data
\end{description}

Each of those has two functions: \verb|compress(audio_data, file_handle)| and \verb|decompress(file_handle, audio_data)|. For compression, it reads the audio data and writes it to the target file. For decompression, it reads the raw data from the file and creates a new \verb|AudioData| object, updating the reference with the decompressed data.

The encoding process has already been described in Section \ref{sec:encoder}, so I will only include specific implementation details here. For the decoding process, just as before, it's doing the same, but inverse operations in the same order, so it will not be elaborated on in detail.

Unless mentioned otherwise, all matrix operations are implemented using \verb|NumPy| and its associated data structures and functions. Writing data to file is done using the \verb|struct.pack| function from the native Python library. During decoding, when the audio data is finalized, it must be converted back to a signed 16-bit integer, so that it can be saved to a WAV file.

\subsection{Utility functions}
To avoid code repetition, a lot of functionality is shared between the various methods using different helper functions. The most important ones are described below.

\subsubsection{Matrix (de)serialization}
There are two core functions for matrix serialization and deserialization to/from a file:

\begin{itemize}
	\item \verb|serialize_matrix(fd, matrix, dtype='f')|
	\item \verb|deserialize_matrix(fd, dtype='f')|
\end{itemize}

For serialization, the input matrix is taken and converted to the target datatype, by default using a 32-bit float, but this can be changed by the caller0. Then, two 32-bit unsigned integers are written representing the amount of rows and columns in the matrix. After that, the matrix is flattened row-wise and each value is written to the file individually.

Deserialization works on the same principle. First we need to read the amount of rows and columns, and then read $rows \times columns \times sizeof(datatype)$ bytes to obtain a data array, which is then re-shaped to the original matrix and returned.

\subsubsection{Array padding}
If we want to run some transformation on an array of values, we may need the array's length to be divisible by a certain number, to be able to split it into equal parts.

There is a utility function for this with the signature \verb|array_pad(array, n)|, where $array$ is the array we want to pad and \verb|n| is the number whose multiple the array's length should be.

We first find the necessary padding by calculating $padding = -length(array) \mod n$ and then add that amount of zeroes to the end of the array, returning both the new array and the amount of padding. The padded zeroes need to be removed again during decoding, otherwise we'll end up with a different signal than the original.

There is a variant of this function for convenience called \verb|array_pad_split(array, n)|, which does the same as the former, but also splits the array into $n$-sized chunks, returning a list of the new subarrays and the padding at the end.

\subsubsection{Matrix splitting}
Similarly to the previous functions, there are times where have a matrix and we want to split it into smaller matrices. There is a function for this called \verb|matrix_split(matrix, n)|.

It takes an input matrix $x \times y$, and splits it into $\left\lfloor \frac{x}{n} \right\rfloor$ submatrices, each sized $n \times y$. There is no padding necessary in this case.

\subsubsection{Range scaling}
There are a few instances where a value (or more often, an entire matrix of values) needs to be scaled within a certain range, e.g. for $\mu$-law companding.

The design part already describes a simple equation to scale a range of values to the $[0,1]$ range (Equation \ref{equ:scale_01}), but the actual implementation allows scaling from an arbitrary range to an arbitrary range. This function is called is \verb|scale_val| and you can see its code in Listing \ref{lst:scale_val}.

\begin{minipage}{\linewidth}
\begin{lstlisting}[caption={Function for scaling value ranges}, label={lst:scale_val}, language=Python]
def scale_val(x, old_min, old_max, new_min, new_max):
	old_range = abs(old_max - old_min)
	new_range = abs(new_max - new_min)
	val = (((x - old_min) / old_range) * new_range) + new_min
	return val
\end{lstlisting}
\end{minipage}

\subsubsection{Uniform quantization}
In ANMF-STFT, it's often needed to uniformly quantize a range of values into $N$ levels. To make this easier, a special class \verb|UniformQuantizer| was implemented. It has three parameters:

\begin{description}
	\item[min\_val] the minimum value in the range
	\item[max\_val] the maximum value in the range
	\item[levels] the desired amount of quantization levels
\end{description}

Internally, it uses a mid-tread quantizer as defined in Equation \ref{equ:midtread_quant}. Rather than returning the quantized value itself, it returns an index in the range $[0, levels-1]$.

This means, that given a \verb|UniformQuantizer| and an index of a quantized value, it can trivially determine the actual quantized value $x$ as:

\begin{align}
x = min\_val + index \cdot step
\end{align}

Where step is defined as:

\begin{align}
step = \frac{max\_val - min\_val}{levels - 1}
\end{align}

It has two methods: \verb|quantize_value(value)| and \verb|dequantize_index(index)|, depending on what you want to do.

\subsubsection{NMF helpers}
This set of functions helps use NMF by doing any processing necessary both before and after and serves as a sort of interface between the codec and NMF. It includes three functions:

\begin{itemize}
	\item \verb|nmf_matrix(matrix, max_iter, rank)|
	\item \verb|nmf_matrix_original(W, H, min_val)|
	\item \verb|increment_by_min(matrix)|
\end{itemize}

\verb|increment_by_min| takes an input matrix, finds $min\_val = |\min(matrix)|$, and increments every element in the matrix by $min\_val$, ensuring that the matrix does not contain any negative values.

\verb|nmf_matrix| takes an input matrix, applies \verb|increment_by_min| on it and then calls NMF, obtaining the weight and coefficient matrices. It then returns both of those matrices along with the minimum value it was initially incremented by.

\verb|nmf_matrix_original| takes the decomposition matrices $W$ and $H$ along with the minimum value $min\_val$, multiplies the two matrices and subtracts $min\_val$ from each element of the new matrix, to ensure the original range of values is preserved. It then returns the final approximated matrix.

\subsection{NMF implementation}
Basic NMF has been implemented in a class called \verb|NMF|. It's a basic implementation for approximating a matrix $V$ as a product of two matrices $WH$.

The class constructor takes six different arguments:

\begin{enumerate}
	\item input matrix
	\item maximum number of iterations
	\item approximation rank
	\item initialization function
	\item cost function
	\item update function
\end{enumerate}

There are a few important functions:

\begin{itemize}
\item The \verb|validate()| ensures that the input matrix does not contain any negative values, raising an exception otherwise.
\item Cost function evaluation is done by the \verb|eval_cost()| function.
\item \verb|factorize()| stands at the core of the class as it does the actual factorization, returning the factors $W$ and $H$. Please refer to Listing \ref{lst:nmf_fact} for the source code.
\end{itemize}

In its default settings, it uses the Euclidean cost function (Equation \ref{equ:cost_euc}) and Euclidean updates (Equation \ref{equ:update_euc}), with the initial elements of $W$ and $H$ being randomly initialized from the range $[0, \max(V)]$.

\begin{minipage}{\linewidth}
\begin{lstlisting}[caption={Python code for the factorize function}, language=Python, label={lst:nmf_fact}]
def factorize(self):
	self.initialize(self)
	last_cost = None
	for i in range(self.max_iter):
		self.update(self)
		cost = self.eval_cost(self)
		if cost == last_cost:
		break
	last_cost = cost
	return self.W, self.H
\end{lstlisting}
\end{minipage}

\subsection{ANMF-RAW}
In my implementation of the ANMF-RAW method, there are three parameters that can be changed:

\begin{description}
	\item[CHUNK\_SHAPE] a 2-tuple containing the number of rows and columns that specifies the shape of each chunk before NMF is applied
	\item[NMF\_MAX\_ITER] maximum number of iterations of NMF before the algorithm is stopped
	\item[NMF\_RANK] target rank of the NMF approximation
\end{description}

If \verb|CHUNK_SHAPE| is set to \verb|None|, the entire array will be turned into a square matrix of dimensions $\left\lceil\sqrt{N}\right\rceil \times \left\lceil\sqrt{N}\right\rceil$, where $N$ is the amount of samples in the audio signal for the given channel. However, this generally leads to a huge loss in quality and it is therefore recommended to always specify a shape tuple.

I use the utility function \verb|array_pad_split| to first split the samples into equal parts of size corresponding to the shape of the desired matrix. Then, each of these parts is re-shaped into the proper shape using \verb|numpy.reshape|, and the values are converted from signed 16-bit integer to a signed 32-bit integer to be able to increment the values in it. The amount of chunks along with the extra padding at the end is then written to the output.

For each submatrix from the original sample array, we apply NMF using the utility function \verb|nmf_matrix| to calculate the NMF, and then write the minimum value to the file along with the serialized matrices $W$ and $H$.

\subsection{ANMF-MDCT}
ANMF-MDCT uses the following parameters:

\begin{description}
	\item[FRAME\_SIZE] how many samples a frame should contain, determines frequency resolution, must be an even number
	\item[NMF\_CHUNK\_SIZE] number that specifies the number of frames in each submatrix before NMF is applied
	\item[NMF\_MAX\_ITER] maximum number of iterations of NMF before the algorithm is stopped
	\item[NMF\_RANK] target rank of the NMF approximation
\end{description}

As no easy to use library was available, I had to implement MDCT on my own.

The signal needs to first be padded to block size using \verb|array_pad|. Before we can apply MDCT, the signal is windowed using the MLT window, i.e. in every frame of size $N = FRAME\_SIZE$, we apply Equation \ref{equ:mlt} to every sample.

In the implementation, there are two different algorithms for calculating the MDCT of the input signal. One is called \verb|mdct_slow| and the other \verb|mdct_fast|.

The slow variant is a direct implementation of Equation \ref{equ:mdct} with the inverse following Equation \ref{equ:imdct}. Unfortunately, as the name suggests, it is rather slow for use in production and only served as a reference for unit testing the fast algorithm.

The fast variant instead relies on two things:

\begin{enumerate}
	\item MDCT can be directly derived from a Discrete cosine transform (DCT) \cite{Babu2013FastAE}
	\item DCT can be calculated quickly with the help of fast Fourier transform (FFT) \cite{makhoul_1980}
\end{enumerate}

Specifically, MDCT can be derived from a type 4 DCT (DCT-IV). We take each block of $FRAME\_SIZE$ samples and split it into four equal parts $(a, b, c, d)$. Then, due to the boundary conditions, the MDCT of this block is equal to a DCT-IV of the inputs $(-c_R-d, a-b_R)$, where $x_R$ refers to $x$ in reverse order. To calculate the DCT-IV of the resulting array, we use the function \verb|scipy.fftpack.dct(input, type=4)|. We normalize the resulting array by dividing every element in it by $2$, making its output equal to the slow algorithm. The function finishes by returning the MDCT coefficient matrix along with the length of padding at the end of the array.

Once we have our MDCT, the resulting MDCT matrix is split using \verb|matrix_split(mdct, NMF_CHUNK_SIZE)|. We write the length of the padding followed by the amount of the split submatrices as 32-bit unsigned integers.

Then, for each MDCT submatrix, we apply the utility function \verb|nmf_matrix|, write the minimum value and then serialize both the matrices.

The decoding process is fairly straightforward other than inverting the MDCT, so that will be described next.

If we want to get the inverse MDCT (IMDCT), it gets a bit tricky. We first run inverse DCT-IV by using \verb|scipy.fftpack.idct(input, type=4)|, giving us back the array in the form of $(-c_R-d, a-b_R)$, which we need to turn back into $(a, b, c, d)$. By cleverly inverting, reversing and concatenating these sub-arrays, we are able to obtain an array in the form of $(a-b_R, b-a_R, c+d_R, d+c_R)$ (notice that some of the values are redundant). We apply the window again and multiply all the elements by two (to cancel out the division in the MDCT) and obtain nearly the original array. As the final step, we need to cancel out the "extra" values such as $-b_R$. To do that, we take two adjacent overlapping blocks and add them together, leaving us with the original array $(a, b, c, d)$.

\subsection{ANMF-STFT}
ANMF-STFT uses the same parameters as ANMF-MDCT and some additional ones:

\begin{description}
	\item[FRAME\_SIZE] how many samples a frame should contain, determines frequency resolution, must be an even number
	\item[NMF\_CHUNK\_SIZE] number that specifies the number of frames in each submatrix before NMF is applied
	\item[NMF\_MAX\_ITER] maximum number of iterations of NMF before the algorithm is stopped
	\item[NMF\_RANK] target rank of the NMF approximation
	\item[MU\_LAW\_W] parameter $\mu$ for $\mu$-law companding of the weight matrix $W$
	\item[MU\_LAW\_H] parameter $\mu$ for $\mu$-law companding of the coefficient matrix $H$
\end{description}

\verb|SciPy| contains a function for the entire STFT calculation, called \verb|scipy.signal.stft|. It handles windowing, overlapping and padding by itself, so it is very handy.

At its core, it uses the fast Fourier transform \cite{Oppenheim:2009:DSP:1795494}. We choose an overlap of 50\% by putting the number of data points per segment to $FRAME\_SIZE$ and the number of overlapping points to $\frac{FRAME\_SIZE}{2}$. For the window function, we choose the Hann window (Section \ref{sec:hann}).

Before further processing, the output STFT matrix is transposed (for the shape to be consistent with the matrix from MDCT), and then we find the phase matrix and the magnitude matrix by calling \verb|numpy.angle| and \verb|numpy.absolute|, respectively, on every element of the matrix.

The magnitude matrix is split into chunks of size $NMF\_CHUNK\_SIZE$ using \verb|matrix_split|, while the phase matrix is quantized using a \verb|UniformQuantizer| in range $[-\pi, \pi]$ with 8 quantization levels and this matrix of quantized values is then Huffman encoded.

Several things are then written to the file in order: the amount of magnitude submatrices (uint32), number of rows in the phase matrix (uint32), length of the Huffman encoded phase matrix (uint32) and the Huffman encoded phase matrix bytes themselves.

The magnitude submatrices are then processed in order. First we use basic NMF via \verb|nmf_matrix|, then scale both the decomposition matrices to the $[0, 1]$ range using \verb|scale_val|, and then $\mu$-law compress both using their respective $MU\_LAW\_[W|H]$ parameter as per Equation \ref{equ:mu_law_compress}.

For matrix $W$, we use $scale_val$ to scale it to range $[0, 2^{32}]$ and change its datatype to unsigned 32-bit integers.

In the case of matrix $H$, it is uniformly quantized in the range $[0, 1]$ with 32 quantization levels, and then similar to the phase matrix, it's Huffman encoded using its associated Huffman table.

Then, for each submatrix, in order, we write to the file: the value we need to subtract after multiplying $WH$ (float64), the original minimum and maximum of the matrix before it was scaled to $[0, 1]$ (float64), the scaled 32-bit integer matrix $W$ using \verb|serialize_matrix|, and finally the encoded matrix $H$ in the same manner as the phase matrix.

When decoding, the process is again the same but in reverse order. To invert the STFT, we call \verb|scipy.signal.istft| using the same parameters as before. It estimates the original signal by using the algorithm described in \cite{griffin_1984}. For this algorithm to work properly and attain perfect reconstruction of the original signal, the COLA constraint (Constant OverLap Add) \cite{bors_2012} must be met. Using a Hann window at 50\% overlap guarantees the satisfaction of this constraint.


\chapter{Evaluation}
In this section you will find an evaluation of the different algorithms, including experiments with various parameters for them and comparison to existing solutions from Section \ref{sec:stateoftheart}.

\section{Methodology}
Audio compression quality is most often evaluated using a series of listening tests (such as the one pictured in Figure \ref{fig:opus_listening_test}). However, due to a lack of resources to conduct such a thing, a different method must be used - although they are generally not as accurate.

One of the most used methods for objectively evaluating perceptible audio quality is PEAQ \cite{peaq_2006}, however its specifications are insufficient, and its implementations proprietary. The exact algorithm and parameters aren't known for the most part. It comes in two variants:

\begin{description}
	\item[PEAQ Basic] intended for real-time use
	\item[PEAQ Advanced] a more comprehensive model, intended for non real-time use
\end{description}

Even though neither of these is directly available, luckily, there are a few open-source alternatives that try to implement similar algorithms, with their quality measured by comparing their output to PEAQ, and by proxy, to listening tests.

One of the more prominent open-source solutions is GstPEAQ, which according to its paper \cite{gstpeaq_paper} performs better than the other implementations. So while it does not conform to the PEAQ recommendation directly, its results are within an acceptable margin and thus this thesis will use GstPEAQ for quality evaluation.

\section{GstPEAQ}
GstPEAQ is a plugin for GStreamer \cite{gstreamer_2016} (a pipeline-based multimedia framework) and its source code is freely available at \cite{gstpeaq_impl}. It implements both the Basic and Advanced mode of PEAQ as specified in \cite{peaq_2006}, however as the standard is under-specified, educated guesses must be taken at points.

And just like PEAQ, the algorithm's main output is a value known as \emph{Objective Difference Grade} (ODG), which evaluates the perceptible impairment (quality difference) between the provided audio and the reference audio. It uses various psychoacoustic "features" of the signal to determine the grade, the details of which won't be covered here - please refer to either paper for specifics.

The ODG scale contains real values from $0$ to $-4$, ranging from imperceptible difference to very annoying for the human ear. Please refer to Table \ref{tab:odg_scale} to see the full scale.

\begin{table}[htbp]\caption{PEAQ - Objective Difference Grade table}
	\label{tab:odg_scale}
	\centering
	\begin{tabular}{|c|l|}
		\hline
		ODG & Impairment description \\ \hline
		$0.0$ & Imperceptible \\
		$-1.0$ & Perceptible, but not annoying \\
		$-2.0$ & Slightly annoying \\
		$-3.0$ & Annoying \\
		$-4.0$ & Very annoying \\
		\hline
	\end{tabular}
\end{table}

\section{Evaluating results}
In this section we will experiment with various parameters for all the encoding methods, find a good compromise between bitrate and ODG, and then compare those results to the reference codings (MP3, Opus) at similar bitrates.

The evaluating process will work in these steps:

\begin{enumerate}
	\item compress example WAV file using ANMF
	\item decompress ANMF back to a WAV file
	\item measure ODG between old and new WAV file
\end{enumerate}

In the case of MP3 or Opus files, they will be decoded to WAV for comparison.

\subsection{Audio examples}
There's a total of four example audio files in WAV format that will be used for testing. Please refer to Table \ref{tab:audio_examples} for a list. All of the examples are $\sim10$ second excerpts from audio files using 44.1 kHz sampling rate and 16-bit signed integer samples. These files can be found in the \verb|examples| folder of the implementation.

\begin{table}[htbp]\caption{Audio example summary}
	\label{tab:audio_examples}
	\centering
	\begin{tabular}{|c|c|l|}
		\hline
		ID & File name & Description \\ \hline
		$00$ & \verb|piano16t.wav| & Clear sounding piano sounds \\
		$01$ & \verb|henry16t.wav| & Average quality English voice \\
		$02$ & \verb|swave16t.wav| & Simple electronic music \\
		$03$ & \verb|taleena16t.wav| & Complex music including lyrics \\
		\hline
	\end{tabular}
\end{table}

\subsection{ANMF-RAW}
.. TODO experiment with parameters and calculate bitrate ..

\subsection{ANMF-MDCT}
.. TODO experiment with parameters and calculate bitrate ..

\subsection{ANMF-STFT}
.. TODO experiment with parameters and calculate bitrate ..

\subsection{Comparison}
.. TODO compare the "best" from the previous 3 vs MP3/Opus ..


\chapter{Conclusion}
The point of this thesis was to design and implement an audio codec with the ultimate goal of audio compression. While this goal was technically met, it leaves much to be desired.

In total, three different methods were designed and implemented:

\begin{description}
	\item[ANMF-RAW] compresses audio using NMF in the time domain directly
	\item[ANMF-MDCT] compresses audio using NMF in the frequency domain via modified discrete cosine transform
	\item[ANMF-STFT] compresses audio using NMF in the frequency domain via short-time Fourier transform
\end{description}

Out of these three methods, only ANMF-STFT ended up being usable in practice. However, similarly to the research it was inspired by, I ended up having a large portion of data in the form of a STFT phase matrix that wasn't feasibly compressible given current knowledge of this particular topic. NMF is ultimately only a small part of the compression and improving the NMF methodology won't reduce the filesize by much.

In the case of both ANMF-RAW and ANMF-MDCT, the data is far too fragile to be approximated without carefully crafted constraints, in which case a proper quantization (similar to MP3 and Opus) will yield far more reliable results than an approximation, i.e. NMF.

The main output of this work is therefore a command line tool capable of compressing arbitrary audio in WAV format to a third of its size, but it is still no match for contemporary codecs such as MP3 or Opus, which are capable of reducing the bitrate by a factor of up to twelve, with far less noticeable quality loss.

In its current form, it is not applicable to real-time use, but if the file structure is altered somewhat, it's definitely possible. It would require the data from each channel to be interweaved similar to a WAV file, so that it can be read sequentially as a stream of frames. Decompression is not a problem, as that boils down to simple matrix multiplication for the most part, which is not an expensive process given today's technology.

One thing that wasn't explored in-depth in terms of implementation was the application of psychoacoustics, to further trim "unnecessary" sounds contained in the signal. While this is most likely feasible and would save some bytes, there is already a noticeable quality loss even when trying to preserve as much information as possible, and as such application of psychoacoustics would not make this codec any more competitive to its counterparts, which are able to compress audio without any apparent distortion.

Considering the low amount of works exploring this particular topic, I believe this thesis will serve as a decent reference for when somebody else may stumble across the same problem. It's also worth mentioning that this might be the first public implementation for this specific problem, and as such it might help with the understanding of the methodology behind this process and possibly lead to improvement in the future.


\bibliographystyle{iso690}
\bibliography{mybibliographyfile}

\appendix

\chapter{List of Symbols}
\begin{tabular}{r l}
	$t$ & symbol representing a time value in seconds \\
	$\tau$ & symbol representing a "slow" time, time index with a lower resolution than $t$ \\
	$\xi$ & symbol representing a frequency value in hertz \\
	$F_s$ & symbol representing a sampling rate of an audio signal \\
	$z$ & symbol representing a complex number \\
	$\phi(z)$ & symbol representing the phase of a complex number \\
	$|z|$ & symbol representing the magnitude of a complex number \\
	$Q(x)$ & function representing a uniform quantizer \\
	$\Delta$ & symbol representing the step size of a uniform quantizer \\
	$F(x)$ & function representing $\mu$-law compression \\
	$x(t)$ & function representing the amplitude of a continuous signal at a time $t$ \\
	$x_n$ & sequence representing the amplitude of a discrete signal indexed by $n$ \\
	$w(t)$ & continuous windowing function at a time $t$ \\
	$w_n$ & discrete windowing function indexed by $n$ \\
	$X(\xi)$ & function representing the frequency component of a signal for a frequency $\xi$ \\
	$S(\xi)$ & the Fourier transform of a continuous signal \\
	$S_k$ & the discrete Fourier transform of a discrete signal \\
	$S(\tau, \xi)$ & the short-time Fourier transform of a continuous signal \\
	$S_{k, \xi}$ & the discrete short-time Fourier transform of a discrete signal \\
	$M_k$ & the Modified discrete cosine transform of a discrete signal \\
	$r$ & rank of a NMF approximation \\
	$S_f$ & frame size \\
	$S_c$ & chunk size
\end{tabular}

\chapter{Abbreviations}
\begin{description}
	\item[NMF] non-negative matrix factorization
	\item[DFT] discrete Fourier transform
	\item[STFT] short-time Fourier transform
	\item[FFT] Fast Fourier Transform
	\item[DCT] discrete cosine transform
	\item[MDCT] modified discrete cosine transform
\end{description}


\chapter{Contents of enclosed CD}

\begin{figure}
	\dirtree{%
		.1 README.txt\DTcomment{the file with CD contents description}.
		.1 src\DTcomment{the directory of source codes}.
		.2 audionmf\DTcomment{implementation sources}.
		.2 thesis\DTcomment{the directory of \LaTeX{} source codes of the thesis}.
		.1 text\DTcomment{the thesis text directory}.
		.2 thesis.pdf\DTcomment{the thesis text in PDF format}.
	}
\end{figure}

\end{document}
