\documentclass[thesis=M,english,hidelinks]{FITthesis}[2018/06/01]

\usepackage[utf8]{inputenc}

\usepackage{graphicx}

\usepackage{dirtree}

\department{Department of Theoretical Computer Science}
\title{Fine-tuning LLVM IR transformation passes}
\authorGN{Tomáš} %author's given name/names
\authorFN{Drbota} %author's surname
\author{Tomáš Drbota} %author's name without academic degrees
\authorWithDegrees{Bc. Tomáš Drbota} %author's name with academic degrees
\supervisor{doc. Ing. Ivan Šimeček, Ph.D.}
\acknowledgements{TODO}
\abstractEN{Modern compilers provide a wide range of different optimizations in an effort to increase the performance of the resulting program. Many of them already offer ways to customize the amount of optimization - for example -O flags in GCC, compiler arguments (e.g. -finline-functions in Clang) or in-line attributes (e.g. \#pragma unroll), but the overall granularity of these options is low.
	
The purpose of this work is to investigate options for further customizing the execution of transformation passes in an optimizing compiler and comparing the results to automatically optimized code. The intent is to ensure there are ways for the user to be able to specifically declare which transformations should or should not be used in a given scope. Implementation is done using LLVM and its C/C++ frontend, Clang.}
\abstractCS{TODO}
\placeForDeclarationOfAuthenticity{Prague}
\keywordsCS{Replace with comma-separated list of keywords in Czech.}
\keywordsEN{llvm, clang, compiler, transformation, optimization}
\declarationOfAuthenticityOption{5}
% \website{http://site.example/thesis} %optional thesis URL


\begin{document}

\setsecnumdepth{part}
\chapter{Introduction}
TODO

\setsecnumdepth{all}
\chapter{LLVM architecture}
\cite{Lattner:MSThesis02}

\section{Intermediate Representation}


\chapter{Clang frontend}

\chapter{Optimizations using LLVM}
\section{Analysis passes}

\section{Transformation passes}

\section{Basic optimizations}

\section{Loop optimizations}
\subsection{Loop unrolling}

\section{Advanced loop optimizations using Polly}
\subsection{Loop tiling}

\section{External optimizations using opt tool}

\chapter{Fine-tuning transformations, implementation}

\chapter{Result evaluation}

\setsecnumdepth{part}
\chapter{Conclusion}


\bibliographystyle{iso690}
\bibliography{mybibliographyfile}

\setsecnumdepth{all}
\appendix

\chapter{Acronyms}
\begin{description}
	\item[todo] TODO
\end{description}


\chapter{Contents of enclosed CD}

\begin{figure}
	\dirtree{%
		.1 readme.txt\DTcomment{the file with CD contents description}.
		.1 exe\DTcomment{the directory with executables}.
		.1 src\DTcomment{the directory of source codes}.
		.2 wbdcm\DTcomment{implementation sources}.
		.2 thesis\DTcomment{the directory of \LaTeX{} source codes of the thesis}.
		.1 text\DTcomment{the thesis text directory}.
		.2 thesis.pdf\DTcomment{the thesis text in PDF format}.
		.2 thesis.ps\DTcomment{the thesis text in PS format}.
	}
\end{figure}

\end{document}
