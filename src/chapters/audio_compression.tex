Compression can be split into two kinds - lossy and lossless. Using "lossless" in the context of audio is a bit misleading, since sampling itself is a lossy process, but using a high enough sampling rate, we will not notice any difference, so sampled audio without any lossy compression will be our baseline.

For audio, lossless compression generally means taking some form of digital audio representation and losslessly compressing this data. This will preserve the signal in its entirety with a reduced bit-rate. However, due to size of such audio (an audio CD could only fit about 80 minutes of such music sampled at 44.1 kHz), it's become more common to use a lossy format.

Lossy compression implies that there will be loss of data, and while this is true, thanks to the application of various psychoacoustic principles size of audio can be greatly reduced without altering human perception, leading to vastly smaller bit-rates for no real cost.

This work focuses on lossy audio compression, therefore only lossy codecs will be considered for comparison.

\section{State of the art}
Due to its qualities of efficiently compacting energy and mitigating artifacts at block boundaries, MDCT is the most commonly used transformation in modern lossy audio coding, and is employed in the most popular audio formats including MP3, Opus, Vorbis or AAC.

In this section, I will elaborate on some of the more popular ones to get an idea of what considerations go into writing a modern audio codec.

\subsection{MP3}
.. TODO diagram ..

MP3, or MPEG-1 Layer III has been standardized in 1991 and has since become widespread throughout a multitude of electronic devices as the de-facto standard for music storage.

It's a very powerful compression/decompression scheme capable of reducing the bit-rate of an audio stream by up to a factor of 12 without any noticeable (to humans) quality degradation. In other words, to transmit CD quality audio, it needs a bitrate of 128 kbps. \cite{Raissi2002TheTB}

The core of MP3 compression is the Modified discrete cosine transform. The signal represented in its PCM form is first split into 32 subbands using an analysis polyphase filterbank, and each of those is further split into 18 MDCT bins, so overall we end up with 576 MDCT frequency bins per frame.

These bins are then sorted into 22 scalefactor bands, which roughly correlate to the 24 bands of human hearing. The point of these bands is that you may individually scale each of them up or down depending on how much precision you need for that specific frequency range. This usually done by dividing and rounding the values in the band, losing a certain amount of information; this process will be reversed during decoding.

The signal is also analyzed using the Fourier transformation, which gives us frequency information for the signal in the same frame, and we can use this information to determine how much to scale each scalefactor band - e.g. if there's some weak sound that will be masked by another, we can assign the band it's in lower precision, saving data. \cite{wilburn_2007}

Once we have the scaled and quantized data, we use the Huffman encoding to losslessly compress these values, and format this output into the final bitstream, encoding our audio.

\subsection{Opus (CELT)}
.. TODO diagram ..

The Opus codec has been standardized fairly recently, in 2012 \cite{rfc6716} compared to other widespread audio codecs. Opus was created from two core technologies - Skype's SILK codec based on linear prediction, and Xiph.Org's CELT codec, based on the MDCT. \cite{opus_celt}

As a result, Opus is capable of performing in three different modes:

\begin{description}
	\item[SILK] used for speech signals
	\item[CELT] used for high-bitrate speech and music
	\item[Hybrid] both SILK and CELT used simultaneously
\end{description}

This thesis will mainly focus on Opus CELT due to it being more general purpose than SILK, which will make it easier to use for comparison with a NMF-based codec. SILK uses a technique known as Linear predictive coding, whose complexity and difficulty of implementation exceeds the scope of this work.

Opus is designed with real-time constraints in mind, that is, for example network music performances which require very low delays. Despite that, however, its compression is comparable to codecs with higher delays intended for streaming or storage. This makes it a good candidate to test against in this work, alongside MP3.

CELT (Constrained Energy Lapped Transform) is based on MDCT similar to MP3, but the main difference is that Opus uses specifically sized bands with ranges similar to the Bark scale in order to preserve the spectral envelope of the signal.

Similar to MP3, Opus CELT works on the basis of quantizing MDCT coefficients, but utilizes various kinds of prediction and focuses more on handling transients. Through various optimizations Opus achieves similar results but at a reduced bitrate.
.. TODO source listening test ..

\subsection{Opus (SILK)}
.. TODO if I have time ..
